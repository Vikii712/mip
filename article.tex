% Metódy inžinierskej práce

\documentclass[10pt,a4paper,]{article}

\usepackage[slovak]{babel}
%\usepackage[T1]{fontenc}
\usepackage[IL2]{fontenc} % lepšia sadzba písmena Ľ než v T1
\usepackage[utf8]{inputenc}
\usepackage{graphicx, subcaption, multicol}
\usepackage{url} % príkaz \url na formátovanie URL
\usepackage{hyperref} % odkazy v texte budú aktívne (pri niektorých triedach dokumentov spôsobuje posun textu)

\usepackage{cite}
%\usepackage{times}

\pagestyle{headings}

\title{{Browser fingerprint and privacy issues regarding data collection from search engines\thanks{Semestrálny projekt v predmete Metódy inžinierskej práce, ak. rok 2023/24, vedenie: Ing. Mohammad Yusuf Momand, MSc.}} % meno a priezvisko vyučujúceho na cvičeniach

\author{Viktória Latičová\\[2pt]
	{\small Slovenská technická univerzita v Bratislave}\\
	{\small Fakulta informatiky a informačných technológií}\\
	{\small \texttt{xlaticova@stuba.sk}}
	}

\date{\small 12. december 2023} % upravte



\begin{document}
\maketitle

\begin{abstract}
This article aims to reflect on some key privacy concerns caused by search engines collecting data while searching on the internet. Besides the obvious task of searching by keywords, they also collect user data to create detailed profiles. Browser fingerprinting techniques identify users uniquely, allowing cross-site tracking leading to location tracking, privacy invasion, or third-party sharing. Protecting users' privacy requires not only technological solutions- because completely removing our browser fingerprint would make searching impossible, but also legal regulations and raising public awareness and education about this topic.
\end{abstract}


\section{Introduction}
\section{Nejaká časť} 
\section{Iná časť} 
\subsection{Nejaké vysvetlenie}
\subsection{Ešte nejaké vysvetlenie}
\section{Dôležitá časť} 
\section{Ešte dôležitejšia časť} 
\section{Záver} 



%\acknowledgement{Ak niekomu chcete poďakovať\ldots}

% týmto sa generuje zoznam literatúry z obsahu súboru literatura.bib podľa toho, na čo sa v článku odkazujete
\bibliography{ref}
\bibliographystyle{plain} % prípadne alpha, abbrv alebo hociktorý iný
\end{document}
