% Metódy inžinierskej práce

\documentclass[10pt,a4paper,column]{article}
\usepackage[top=3cm, bottom=4cm, left=3cm, right=3cm, marginparwidth=1.75cm]{geometry}
\usepackage[english]{babel}
\usepackage[T1]{fontenc}
\usepackage[IL2]{fontenc} 
\usepackage[utf8]{inputenc}
\usepackage{graphicx, subcaption, multicol}
\usepackage{url} % príkaz \url na formátovanie URL
\usepackage{hyperref} % odkazy v texte budú aktívne (pri niektorých triedach dokumentov spôsobuje posun textu)

\usepackage{cite}
%\usepackage{times}

\pagestyle{headings}

\title{Browser fingerprint and privacy issues regarding data collection from search engines\thanks{Semestrálny projekt v predmete Metódy inžinierskej práce, ak. rok 2023/24, vedenie: Ing. Mohammad Yusuf Momand, MSc.}} % meno a priezvisko vyučujúceho na cvičeniach

\author{Viktória Latičová\\[2pt]
	{\small Slovenská technická univerzita v Bratislave}\\
	{\small Fakulta informatiky a informačných technológií}\\
	{\small \texttt{xlaticova@stuba.sk}}
	}

\date{\small 12. december 2023} % upravte



\begin{document}
\maketitle

\begin{abstract}
This article aims to reflect on some key privacy concerns caused by search engines collecting data while searching on the internet. Besides the obvious task of searching by keywords, they also collect user data
to create detailed profiles. Browser fingerprinting techniques identify users uniquely, allowing cross-site tracking leading to location tracking, privacy invasion, or third-party sharing. Protecting users' privacy requires not only technological solutions- because completely removing our browser fingerprint would make searching impossible, but also legal regulations and raising public awareness and education about this topic.
\end{abstract}


\section{Introduction}
Tracking users across various websites is not a novel concept in the online realm. Websites often monitor users without their awareness, and these practices can serve various purposes. Some are well-intentioned, such as personalization and user interface optimization, while others involve the sale of user information to third-party websites or the more extensive tracking of personal data.
\section{browser fingerprint} 

The most well-known tracking technique to the public is assigning unique identifiers to cookies, which are then stored on their devices for later use. But unlike cookies, creating a browser fingerprint actually does not require the permission of the user and as it is not stored in any device, it does not leave a trace either. 

A browser fingerprint is essentially a user profile created from data collected during online activities. This profile includes a wide range of device-related information, such as IP address, time zone, CPU details, screen resolution, plugins, ad-blockers, and much more.Some webmail servives are even recognized for inspecting emails of users, that have never granted a permission to them.(Web tracking mechanisms) Their unique fingerprint is now a new tool for websides to assign their identity to activities on internet. 

\begin{center}
\includegraphics[scale = 0.4]{Browser-Fingerprint-example.jpg}
\end{center}
\subsection{click tracking - redirect link} 
One of the most commonly used methods for obtaining information from users is quite straightforward – using URLs. Many people remain unaware that major search engines like Google, Yahoo, or Bing permit this kind of click tracking. Tracking clicks on links is not always employed for bad purposes. In most cases, it is used in affiliate marketing by companies for statistical and analytical purposes. 

In certain cases, attackers can create a unique link that initially redirects the user to different site, take information that is needed, and swiftly redirects the user to the destination site. This method can be dangerous because of its almost non-existent chance of being discovered just by looking at the URL. 
\subsection{network and location fingerprinting}
One of the most straightforward features to identify is the IP address of a user. Obtaining a user's location from an IP involves a process known as geolocation. When a user accesses a website, their IP address is captured. An approximate geographic location can be provided by IP geolocation databases of providers. This information can include the country, city, or even latitude and longitude. The presence of a proxy can be easily detected and, with some skill, even bypassed.

At the same time, it is important to note that geolocation is not always 100\% accurate. It only provides an approximation of location. Also, the use of a VPN gateway or a Tor exit relay can be employed to mask the real IP address

\subsection{operating system fingerprinting}
Operating system fingerprinting reffers to a process of detecting operating system remotly when a device is connected to a network. (osf v anta). OS detection helps improve network management, ensures that the software is compatible with users operating system and also can detect unathorised attacks. The architecture of the OS can be recognised by Javascript and flash. They can detect informations from local timezone, language just in miliseconds. Later, they can also detect audio capabilities of the device, the support of printing of the device or even store tracking infromation. (web tracking mechanisms)

We can devide os fingerprinting into two types: passive fingerprinting, where the attacker does not send out any packets to the victims. In this case, the victim connects to the attacker, most of the time by being tricked into conecting in some way. The other possibility is that the attacker just „sits“ between the user and the webside, that the user is trying to connect to and captures information along the way.Active fingerprinting, on the other hand requires a attacker to send specific network requests to the victim in order to be able to determine system information. 

\subsection{browser extentions fingerprinting}
Browser extentinions are programs, that are written to add functionalities to browser.They offer users a ability to block ads, take screenshots, manage passwords, assist when researching on internet. using more extentions extends the tracking powers of atacker, as  users become more and more unique for fingerprinting. Some extentions, without knowlege of the user, can be also designed to collect data from user from the start. These extensions are written using HTML, CSS, and JavaScript, which gives them the capability to track user activity similarly to websites..(extended tracking powers).
\begin{center}
\includegraphics[scale = 0.4]{Entropy.png}
\end{center}
\section{consequences}
With fingerprinting, attackers can find weaknesses in a user's device or browser, which could result in targeted attacks. Users can be uniquely identified and tracked via their browser fingerprints, even if they use private browsing modes or get rid of their cookies. It might be challenging for users to stay anonymous online after being exposed by profiles created with browser fingerprinting. To reduce the uniqueness of their fingerprint, some users may disable or alter browser security measures, which then compromises their online security.
\subsection{hacking}

\section{slicing} 
\section{public awareness} 
\section{reaction to lectures}
\section{conclusion} 



%\acknowledgement{Ak niekomu chcete poďakovať\ldots}

% týmto sa generuje zoznam literatúry z obsahu súboru literatura.bib podľa toho, na čo sa v článku odkazujete
\bibliography{ref}
\bibliographystyle{plain} % prípadne alpha, abbrv alebo hociktorý iný
\end{document}


